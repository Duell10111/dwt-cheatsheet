\documentclass[german]{latex4ei/latex4ei_sheet}

\title{Example\\ Cheat Sheet}
\author{Konstantin Späth}					% optional, delete if unchanged
\myemail{duell10111@t-online.de}	

\begin{document}

\section{Grundbegriffe Wahrscheinlichkeit}

\begin{sectionbox}
	\subsection{Mengen- und Boolsche Algebra}
	\begin{tablebox}{lll}
		%& $(P(\Omega);\capdot , \cupplus, \overline{A};\Omega,\emptyset )$\\ \mrule
		Kommutativ 		& $A \capdot B = B \capdot A$ & $A \cupplus B = B \cupplus A$\\
		Assoziativ 		& \multicolumn{2}{l}{ $(A \capdot B) \capdot C = A \capdot (B \capdot C)$} \\
		& \multicolumn{2}{l}{$(A \cupplus B) \cupplus C = A \cupplus (B \cupplus C)$} \\
		Distributiv 	& \multicolumn{2}{l}{$A \capdot (B \cupplus C) = (A \capdot B) \cupplus (A \capdot C)$}\\
		& \multicolumn{2}{l}{ $A \cupplus (B \capdot C) = (A \cupplus B) \capdot (A \cupplus C)$}\\ \cmrule
		Indempotenz		& $A \capdot A = A$ & $A \cupplus A = A$\\
		Absorbtion		& $A \capdot (A \cupplus B) = A$ & $A \cupplus (A \capdot B) = A$\\
		Neutralität		& $A \capdot \Omega = A$ & $A \cupplus \emptyset = A$\\
		Dominant		& $A \capdot \emptyset = \emptyset$ & $A \cupplus \Omega = \Omega$\\
		Komplement		& $A \capdot \overline{A} = \emptyset$ & $A \cupplus \overline{A} = \Omega$\\
		& $\overline{\overline{A}} = A$ & $\ol{\Omega} = \emptyset$\\
		De Morgan		& $\overline{A \capdot B} = \overline{A} \cupplus \overline{B}$ & $\overline{A \cupplus B} = \overline{A} \capdot \overline{B}$\\
	\end{tablebox}
\end{sectionbox}
	
\begin{sectionbox}
	\subsection{Kombinatorik}
	Mögliche Variationen/Kombinationen um $k$ Elemente von maximal $n$ Elementen zu wählen bzw. $k$ Elemente auf $n$ Felder zu verteilen:
	\begin{tablebox}{l|cc}
		& \large Mit Reihenfolge & \large Reihenfolge egal\\ \cmrule
		%& ungleiche Elemente & gleiche Elemente \\
		\large Mit Wiederholung & \large $n^k$ & \Large $\binom{n+k-1}{k}$\\[0.2em]
		\large Ohne Wiederholung & \Large $\frac{n!}{(n-k)!}$ & \Large $\binom nk$\\
	\end{tablebox}
	Permutation von $n$ mit jeweils $k$ gleichen Elementen: $\frac{n!}{k_1 ! \cdot k_2 ! \cdot ...}$ \\
	$\binom nk = \binom n{n-k} = \frac{n!}{k! \cdot (n-k)!}$ \quad $\binom 42 = 6$ \quad $\binom 52 = 10$
\end{sectionbox}

\begin{sectionbox}
	\subsection{Grundbegriffe}

	\begin{tablebox}{ll}
		Tupel & $(i,j) \neq (j,i)$ für $i \neq j$ \\
		Ungeordnetes Paar & $\{i,j\} = \{j,i\}$ \\
		Potenzmenge & $\mathbb \P(\Omega)$ ist Menge aller Teilmengen von $\Omega$ \\
	\end{tablebox}
\end{sectionbox}

\begin{sectionbox}
	\subsection{Integralarten}

	\renewcommand{\arraystretch}{1.6}
	\begin{tablebox}{ccc}
		$F(x)$ & $f(x)$ & $f'(x)$ \\ \cmrule
		$\frac{1}{q+1}x^{q+1}$ & $x^q$ & $qx^{q-1}$ \\
		\raisebox{-0.2em}{$\frac{2\sqrt{ax^3}}{3}$} & $\sqrt{ax}$ & \raisebox{0.2em}{$\frac{a}{2\sqrt{ax}}$}\\
		$x\ln(ax) -x$ & $\ln(ax)$ & $\textstyle \frac{1}{x}$\\
		%e^x & e^x & e^x \\
		$\frac{1}{a^2} e^{ax}(ax- 1)$ & $x \cdot e^{ax}$ & $e^{ax}(ax+1)$ \\
		$\frac{a^x}{\ln(a)}$ & $a^x$ & $a^x \ln(a)$ \\
	\end{tablebox}
	\vspace{-8pt}
	\begin{tablebox}{ll}
		$\int \frac{\diff t}{\sqrt{at+b}} = \frac{2 \sqrt{at+b}}{a}$ & $\int t^2 e^{at} \diff t = \frac{(ax-1)^2+1}{a^3} e^{at}$\\
		$\int t e^{at} \diff t = \frac{at-1}{a^2} e^{at}$ & $\int x e^{ax^2} \diff x = \frac{1}{2a} e^{ax^2}$\\
	\end{tablebox}
\end{sectionbox}

\begin{sectionbox}
	\subsection{Binome, Trinome}
	$(a\pm b)^2 = a^2 \pm 2ab + b^2$ \hfill $a^2 - b^2 = (a-b)(a+b)$\\
	$(a \pm b)^3 = a^3 \pm 3a^2b + 3ab^2 \pm b^3$\\
	$(a+b+c)^2 = a^2 + b^2 + c^2 + 2ab + 2ac + 2bc$
\end{sectionbox}

% SECTION ====================================================================================
\section{Bedingte Wahrscheinlichkeit und \newline Unabhängigkeit}
% ============================================================================================
%Wechselseite Information $I(A,B) = \log_2 \frac{\P_B (A)}{\P(A)}$\\ % Wichtig für Aufgaben?
%Es gilt: $\P(A \cap B) = \P(A) - \P(A \cap B^\complement)$\\
\begin{sectionbox}
	\subsection{Bedingte Wahrscheinlichkeit}
	Bedingte Wahrscheinlichkeit für $A$ falls $B$ bereits eingetreten ist:\\
	$\P_B(A) = \P(A|B) = \frac{\P(A \cap B)}{\P(B)}$\\ %\qquad\quad $\P(B|A) = \P(A|B) \frac{\P(B)}{\P(A)}$\\

	\subsubsection{Totale Wahrscheinlichkeit und Satz von Bayes}
	Es muss gelten: $\bigcup\limits_{i \in I} B_i = \Omega$ für $B_i \cap B_j = \emptyset$, $\forall i \neq j$ \\
	\begin{tabular}{ll}
		Totale Wahrscheinlichkeit: & $\P(A) = \sum\limits_{i \in I} \P(A|B_i)\P(B_i)$\\
		Satz von Bayes: & $\P(B_k | A) = \frac{\P(A | B_k)\P(B_k)}{\sum\limits_{i \in I} \P(A|B_i) \P(B_i)}$\\
	\end{tabular}


	\subsubsection{Multiplikationssatz}
	$\P(A \cap B) = \P(A|B)\P(B) = \P(B|A)\P(A)$
	\\ \\
	Beliebig viele Ereignisse:\\
	$\P\left(A_1 \cap A_2 \cap \shdots \cap A_k\right) \newline
	= \P\left(A_{\pi(1)}\right)\P\left(A_{\pi(2)}|A_{\pi(1)}\right)\P\left(A_{\pi(3)}|A_{\pi(2)} \cap A_{\pi(1)}\right) \times \newline
	\shdots \times \P\left(A_{\pi(k)}|A_{\pi(k-1)} \cap \shdots \cap A_{\pi(1)}\right)$
\end{sectionbox}

\begin{sectionbox}
	\subsection{Stochastische Unabhängigkeit}
	Ereignise $A$ und $B$ sind unabhängig falls:\\
	$\P(A \cap B) = \P(A)\P(B)$ \\
	$\Rightarrow \P(B|A)=\P(B)$  \\ \\
	\textbf{Allgemein:}  \\
	$\P\left(\bigcap\limits_{i \in J} A_i\right) = \prod\limits_{i \in J}\P\left(A_i\right)$
	mit Indexmenge $I$ und $\emptyset \neq J \subseteq I$
\end{sectionbox}

\section{Zufallsvariablen}
% ============================================================================================
\begin{sectionbox}
	\subsection{Definition}
	$\X : \Omega \mapsto \Omega'$ ist Zufallsvariable, wenn für jedes Ereignis $A' \in \F'$  \\
	im Bildraum ein Ereignis $A$ im Urbildraum $\F$ existiert, \\
	sodass $\left\{\omega \in \Omega|\X(\omega) \in A'\right\} \in \F$\\ \\
\end{sectionbox}

\begin{sectionbox}
	\subsection{Unabhängigkeit von Zufallsvariablen}
	Zufallsvariablen $\X_1,\shdots,\X_n$ sind stochastisch unabhängig, wenn für jedes $\vec{x} = [x_1,\shdots,x_n]^\top \in \R^n$ gilt:
	\[\boxed{\P(\{\X_1 \leq x_1,\shdots,\X_n \leq x_n\}) = \prod\limits_{i=1}^{n}{\P(\{\X_i \leq x_i\})}}\]
	\underline{Gleichbedeutend:}\\
	\begin{tabular}{l}
		$F_{\X_1,\shdots,\X_n}(x_1,\shdots,x_n) = \prod\limits_{i=1}^{n}{F_{\X_i}(x_i)}$\\
		$p_{\X_1,\shdots,\X_n}(x_1,\shdots,x_n) = \prod\limits_{i=1}^{n}{p_{\X_i}(x_i)}$\\
		$f_{\X_1,\shdots,\X_n}(x_1,\shdots,x_n) = \prod\limits_{i=1}^{n}{f_{\X_i}(x_i)}$\\
	\end{tabular}
\end{sectionbox}

%Wechselseitige Information $I(x_i,y_j) = \log_2 \frac{p_{\X|\Y}(x_i|x_j)}{p_{\X}(x)}$\\
%Transinformation $I(\X,\Y)$\\
%Entropie und Transinforation
\begin{sectionbox}
	\subsection{Bedingte Zufallsvariablen}
	Bedingte Wahrscheinlichkeit für Zufallsvariablen:\\
	\begin{tabular}{ll}
		Ereignis A gegeben: & $F_{\X|A}(x|A) = \P\left(\eset{\X \le x} | A\right)$\\
		ZV $\Y$ gegeben: & $F_{\X|\Y}(x|y)= \P\left(\eset{\X \le x} | \eset{\Y = y}\right)$\\
		& $p_{\X|\Y}(x|y) = \frac{p_{\X,\Y}(x,y)}{p_{\Y}(y)}$\\
		& $f_{\X|\Y}(x|y) = \frac{f_{\X,\Y}(x,y)}{f_{\Y}(y)} = \frac{\diff F_{X|Y}(x|y)}{\diff x}$\\
	\end{tabular}

\end{sectionbox}

\section{Kontinuierlich Wahrscheinlichkeitsräume}


\begin{sectionbox}
    \subsection{Allgemeines}
    
    \begin{tablebox}{cc}
    
    Verteilfkt & 
    
    \end{tablebox}
        
    
\end{sectionbox}

\subsection{Kolomogorov-Axiome und $\sigma$-Algebren}

\begin{sectionbox}
    
    \subsubsection{$\sigma$-Algebren}
    Sei $\Omega$ eine Menge. Eine Menge A $\subseteq \mathcal{P}(\Omega)$ heißt $\sigma$-Algebra über $\Omega$, wenn folgende Eigenschaften erfüllt sind:
    
    \begin{enumerate}
        \item $\Omega \in A$
        \item Wenn $B \in A$, dann folgt $\overline{B} \in A$
        \item Für $n \in \N$ sei $C_{n} \in A$. Dann gilt auch $\bigcup_{n=1}^{\infty} C_{n} \in A$
    \end{enumerate}
    
    
\end{sectionbox}

\end{document}